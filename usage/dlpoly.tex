\chapter{\dlpoly interface}
\label{sec:usage:dlpoly}
\textbf{WARNING: The \dlpoly interface is still experimental (in development)
but it does support the Iterative Boltzmann Inversion and Inverse Monte Carlo schemes. 
The Force Matching might work as well, although it has not been tested thoroughly.}

\section{General remarks on using \votca with \dlpoly}

The \dlpoly interface fully supports coarse-grain mapping of a full-atom system 
previuosly simulated with any version of \dlpoly, including \dlpoly-Classic. 
However, the full optimization of the effective potentials with the aid of iterative 
methods will only become possible when the new release of \dlpolyfour (4.06) is made 
public; the reason being the incapability of earlier \dlpoly versions of using 
user-specified tabulated force-fields for intramolecular, aka "bonded", interactions:
bonds, angles, dihedral angles (torsions). Below the coarse-graining and CG force-field 
optimization with the aid of the latest \dlpolyfour version (4.06+) are outlined. 

Running \votca with \dlpolyfour as MD simulation engine is very similar to doing so with \gromacs. 
The three types of required input files in the case of \dlpoly are: CONTROL -- containing 
the simulation directives and parameters (instead of \texttt{.mdp} file for \gromacs), 
FIELD -- the topology and force-field specifications (instead of \texttt{.top} and \texttt{.tpr} 
files), and CONFIG (instead of \texttt{.gro} file) -- the initial configuration file, 
containing the MD cell matrix and particle coordinates (it can also include initial velocities 
and/or forces); for details see \dlpolyfour manual. Most of the \votca tools and scripts described 
above in the case of using \gromacs will work in the same manner, with the following conventional  
substitutions for the (default) file names used in options for \votca scripts, as necessary: 
\begin{verbatim}
.dlpf = the topology read from FIELD or written to FIELD_CGV
.dlpc = the configuration read from CONFIG or written to CONFIG_CGV
.dlph = the trajectory read from HISTORY or written to HISTORY_CGV
\end{verbatim}
It is also possible to specify file names different from the standard \dlpoly convention, 
in which case the user has to use the corresponding dot-preceded extension(s); 
for example: FA-FIELD.dlpf instead of FIELD or CG-HISTORY.dlph instead of HISTORY\_CGV 
(see  section \ref{sec:ref_programs}, as well as the man pages or output of \votca commands, 
with option \progopt{---help}). 

\votca follows the \dlpoly conventions for file names and formats. Thus, \texttt{csg\_dlptopol} 
and \texttt{csg\_map} produce the CG topology (FIELD\_CGV by default), configuration (CONFIG\_CGV), 
and/or trajectory (HISTORY\_CGV) files fully compatible with and usable by \dlpoly. 
{\bf Note that the ability of these tools to read and write a plethora of different file formats 
provides means to convert input and output files between the simulation packages supported 
by \votca, e.g. \gromacs -- \dlpoly or vice versa. The user is, however, strongly advised 
to check the resulting files for consistency before using them).}

Similarly, the distribution analysis and potential/force generation utilities, such as \texttt{csg\_stat} 
and \votca scripts, will read and write \dlpoly-formatted files; in particular, the tabulated 
force-field files containing the potential and force/virial data: TABLE -- for short-range (VdW) 
"non-bonded" interactions, TABBND, TABANG and TABDIH -- for "bonded" interations: bonds, bending 
angles and dihedrals, correspondingly (for the format details see \dlpolyfour manual). Note, however, 
that the latter three files can only be used by \dlpolyfour (4.06+). 

The user is advised to search for "dlpoly" through the \texttt{csg\_defaults.xml}, \texttt{csg\_table} 
files and in scripts located in \texttt{share/votca/scripts/inverse/} in order to find out about 
the xml-tags and options specific for \dlpoly; see also sections~\ref{sec:ref_options} 
and~\ref{sec:csg_table}.


