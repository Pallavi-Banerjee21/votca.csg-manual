\chapter{Force matching}
\begin{figure}
   \centering
   \includegraphics{usage/fig/flow_fmatch.eps}
   \caption{Flowchart to perform force matching.}
\end{figure}
The force matching algorithm with cubic spline basis is implemented in the \prog{csg_fmatch} utility. A list of available options can be found in the reference section of \prog{csg_fmatch} (command \texttt{--h}).

\section{Program input}
\prog{csg_fmatch} needs an atomistic reference run to perform coarse-graining. Therefore, the trajectory file {\em must contain forces } (note that there is a suitable option in the \gromacs ~.mdp file), otherwise \prog{csg_fmatch} will not be able run.

In addition, a mapping scheme has to be created, which defines the coarse-grained model (see \sect{sec:mapping}). At last, a control file has to be created, which contains all the information for coarse-graining the interactions and parameters for the force-matching run. This file is specified by the tag \progopt{--options} in the \xml format. An example might look like the following
\begin{lstlisting}
   <cg>
     <!--fmatch section -->
     <fmatch>
       <!--Number of frames for block averaging -->
       <frames_per_block>6</frames_per_block>
       <!--Constrained least squares?-->
       <constrainedLS>false</constrainedLS>
     </fmatch>
     <!-- example for a non-bonded interaction entry -->
     <non-bonded>
       <!-- name of the interaction -->
       <name>CG-CG</name>
       <type1>A</type1>
       <type2>A</type2>
       <!-- fmatch specific stuff -->
       <fmatch>
         <min>0.27</min>
         <max>1.2</max>
         <step>0.02</step>
         <out_step>0.005</out_step>
       </fmatch>
     </non-bonded>
   </cg>
\end{lstlisting}
A full description of all available options can be found in \sect{sec:ref_options}.

\section{Program output}
\prog{csg_fmatch} produces a separate .force file for each interaction, specified in the CG-options file (option \progopt{options}).
These files have 4 columns containing distance, corresponding force, a table flag and the force error, which is estimated via a block-averaging procedure.
If you are working with an angle, then the first column will contain the corresponding angle in radians.

To get table-files for \gromacs, integrate the forces in order to get potentials and do extrapolation and potentially smoothing afterwards.

Output files are not only produced at the end of the program execution, but also after every successful processing of each block. The user is free to have a look at the output files and decide to stop \prog{csg_fmatch}, provided the force error is small enough.

\section{Integration and extrapolation of .force files }
To convert forces (*.force) to potentials (*.pot), tables have to be integrated. To use the built-in integration command from the scripting framework, execute
\begin{verbatim}
 csg_call table integrate CG-CG.force CG-CG.pot
\end{verbatim}

This command calls the \prog{integrate.pl} script, which integrates the force and writes the potential to the .pot file.

In general, each potential contains regions which are not sampled. However, a special value can be defined in the tabulated potential file for the simulation program. This can be used to never sample very small distances in the case of non-bonded interactions or to extend the file to long ranges. To extrapolate the potential, the grid has to be resampled first as shown below
\begin{verbatim}
 csg_resample --in CH2-CH2.pot --out CH2-CH2.resampled.pot --grid 0:0.001:1.4
\end{verbatim}

Finally, the \prog{extrapolate.pl} script can be called:
\begin{verbatim}
 csg_call table extrapolate --function linear
--region leftright CH2-CH2.resampled.pot CH2-CH2.extrapolated.pot
\end{verbatim}
